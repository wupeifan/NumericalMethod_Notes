% \documentclass[handout]{beamer} % Suppress the pauses to print as handout
\documentclass[aspectratio=169, 11pt]{beamer}
% \documentclass[11pt]{beamer} % Comment the line above and use this line for 4:3 aspect ratio screens
\usepackage{advdate}
\usepackage{amsmath}
\usepackage{amsthm}
\usepackage{amsfonts}
\usepackage{amssymb}
\usepackage{array}
\usepackage{color}
\usepackage{helvet}
\usepackage{hyperref}
\usepackage{mathpazo}
\usepackage{multirow}

\usetheme{CambridgeUS}
\usecolortheme{dolphin}
\usefonttheme[onlymath]{serif}

% Define colors considering color blindness
% Use MATLAB default color palette
\definecolor{blue}{rgb}{0, 0.447, 0.741}
\definecolor{red}{rgb}{0.85, 0.325, 0.098}
\definecolor{yellow}{rgb}{0.929, 0.694, 0.125}
\definecolor{purple}{rgb}{0.494, 0.184, 0.556}
\definecolor{green}{rgb}{0.466, 0.674, 0.188}
\definecolor{cyan}{rgb}{0.301, 0.745, 0.933}
\definecolor{magenta}{rgb}{0.635, 0.078, 0.184}
\definecolor{amber}{rgb}{1.0, 0.75, 0.0}

% Self-defined background and frametitle colors
\definecolor{MyBackground}{RGB}{255, 255, 230}
% Uncomment the following line for plain white background
% \definecolor{MyBackground}{RGB}{255, 255, 255}
\setbeamercolor{background canvas}{bg=MyBackground!40}
\setbeamercolor{frametitle}{bg=MyBackground!80}
% Customize word colors
\setbeamercolor{frametitle}{fg=blue}
\setbeamercolor{title}{fg=blue}
\setbeamercolor{itemize item}{fg=blue}
\setbeamercolor{itemize subitem}{fg=blue}
\setbeamercolor{enumerate item}{fg=blue}
\setbeamercolor{enumerate subitem}{fg=blue}
\setbeamercolor{button}{bg=MyBackground,fg=blue}

% Page number on the bottom
\setbeamertemplate{footline}[frame number]
\setbeamerfont{footline}{size=\fontsize{10}{12}\selectfont}
\setbeamercolor{page number in head/foot}{fg=blue}
% No navigation symbols
\setbeamertemplate{navigation symbols}{}
% No headlines
\setbeamertemplate{headline}{}
% Appendix slides don't count
\newcommand{\backupbegin}{
   \newcounter{framenumberappendix}
   \setcounter{framenumberappendix}{\value{framenumber}}
}
\newcommand{\backupend}{
   \addtocounter{framenumberappendix}{-\value{framenumber}}
   \addtocounter{framenumber}{\value{framenumberappendix}}
}
% Add section title at the beginning of each part
\AtBeginSection[]{
  \begin{frame}[noframenumbering, plain]
  \vfill
  \centering
    \begin{beamercolorbox}[sep=8pt, center, shadow=true, rounded=true]{title}
      \usebeamerfont{frametitle}\insertsectionhead\par%
    \end{beamercolorbox}
  \vfill
  \end{frame}
}

% New theorem environments
\newtheorem{thm}{Theorem}
\newtheorem{lem}[thm]{Lemma}
\newtheorem{prop}[thm]{Proposition}

\beamersetuncovermixins{\opaqueness<1>{25}}{\opaqueness<2->{15}}

% Set pause options to be invisible instead of transparent
\setbeamercovered{invisible}

\begin{document}
\title{Lecture 6 \\ Perturbation Methods}
\author[Wu]{Peifan Wu}
\date{UBC}

\begin{frame}
\titlepage
\end{frame}

\begin{frame}
\frametitle{Local Approximations}
  \begin{itemize}
    \item[--] DSGE models are characterized by \textcolor{blue}{a set of nonlinear equations}
    \bigskip
    \item[--] Use local approximation to transform the nonlinear system into a system of linear equations
    \bigskip
    \item[--] Two requirements: small shocks, no occasionally binding constraints
  \end{itemize}
\end{frame}

\begin{frame}
\frametitle{Perturbing Equations}
  \begin{itemize}
    \item[--] Want: find a solution $y=g\left(x\right)$ for equation system $f\left(x,y\right)=0$
    \item[--] Steady state $f\left(\bar{x},\bar{y}\right)=0$
    \bigskip
    \item[--] (First-order) Taylor expansion of $f$
    \[
      f\left(x,g\left(x\right)\right)\approx f\left(\bar{x},\bar{y}\right)+\frac{\partial f}{\partial x}\left(\bar{x}\right)\left(x-\bar{x}\right)+\frac{\partial f}{\partial y}\left(\bar{y}\right)g'\left(\bar{x}\right)\left(x-\bar{x}\right)
    \]
    \bigskip
    \item[--] Back out $g'\left(\bar{x}\right)$ and approximate the policy function $g\left(x\right)\approx g\left(\bar{x}\right)+\left(x-\bar{x}\right)g'\left(\bar{x}\right)$
  \end{itemize}
\end{frame}

\begin{frame}
\frametitle{Perturbing DSGE}
  \begin{itemize}
    \item[--] Equation system:  $\mathbb{E}_{t}\mathcal{H}\left(y',y,x',x;\sigma\right)=0$
    \item[--] Want: \textcolor{blue}{policy function} $y=g\left(x;\sigma\right)$ and \textcolor{blue}{law of the motions} of the states $x'=h\left(x;\sigma\right)$
    \bigskip
    \item[--] Perturbation parameter is the variance scalar of the shocks, \textcolor{red}{$\sigma$}
    \item[--] There can be multiple uncertainties, but \textcolor{red}{only one perturbation variable}
    \bigskip
    \item[--] Find approximate solution of $g$ and $h$ around $x=\bar{x},\sigma=0$
  \end{itemize}
\end{frame}

\begin{frame}
\frametitle{Perturbing DSGE}
  \begin{itemize}
    \item[--] Plug in the proposed solution
    \[
      F\left(x;\sigma\right)\equiv\mathbb{E}_{t}\mathcal{H}\left(g\left(h\left(x;\sigma\right)\right),g\left(x;\sigma\right),h\left(x;\sigma\right),x;\sigma\right)=0
    \]
    \bigskip
    \item[--] $F=0$ for \textcolor{red}{any} values of $x$ and $\sigma$, so the directives at any direction should be $0$
    \bigskip
    \item[--] In particular, we take the derivatives around $x=\bar{x}, \sigma=0$
    \[
      F_{x^{i}\sigma^{j}}\left(\bar{x};0\right)=0,\forall i,j
    \]
  \end{itemize}
\end{frame}

\begin{frame}
\frametitle{Perturbing DSGE}
  \begin{itemize}
    \item[--] We propose solution of $g,h$
    \begin{align*}
      g\left(x;\sigma\right) & =g\left(\bar{x};0\right)+g_{x}\left(x-\bar{x}\right)+g_{\sigma}\sigma\\
      h\left(x;\sigma\right) & =h\left(\bar{x};0\right)+h_{x}\left(x-\bar{x}\right)+h_{\sigma}\sigma
    \end{align*}
    \bigskip
    \item[--] $g_x,h_x,g_\sigma,h_\sigma$ should solve $F_x=0,F_\sigma=0$
    \begin{align*}
      F_{x} & =\mathcal{H}_{y'}g_{x}h_{x}+\mathcal{H}_{y}g_{x}+\mathcal{H}_{x'}h_{x}+\mathcal{H}_{x}=0\\
      F_{\sigma} & =\mathcal{H}_{y'}g_{x}h_{\sigma}+\mathcal{H}_{y}g_{\sigma}+\mathcal{H}_{y}g_{\sigma}+\mathcal{H}_{x'}h_{\sigma}=0
    \end{align*}
    \bigskip
    \item[--] Observations:
    \begin{itemize}
      \item[1.] The first equation is a linear-quadratic system. Typical solution method applies (Blanchard-Kahn, Klein, Sims \textit{gensys})
      \item[2.] The second equation is homogeneous in $g_{\sigma}$ and $h_{\sigma}$ -- certainty equivalence
    \end{itemize}
  \end{itemize}
\end{frame}

\end{document}
